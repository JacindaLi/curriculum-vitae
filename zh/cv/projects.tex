\section{项目经历}
%-------------------------------------------------------
% \datedsubsection{\textbf{MiniC 编辑器} \href{https://github.com/yanshengjia/nucleon}{[Link]}}{2016.10 --\ 2016.12}
% \textit{独立开发}
% \begin{itemize}
%   \item 用 Python 和 Qt 开发了一个有语法高亮功能的 MiniC 编辑器
%   \item 整合了这个编辑器和 MiniC 编译器以检测代码错误
% \end{itemize}

%-------------------------------------------------------
\datedsubsection{\textbf{Essay Rater}}{\textit{北京, 中国}}
\datedsubsection{\textit{主程序员}}{2017.10 --\ \textit{现今}}
\begin{itemize}
  \item 使用 LSTM 来提取文本特征从而预测作文整体得分, 基于统计模型预测作文各维度得分
  \item 实现了基于 Ngrams 语言模型的拼写修正模块
  \item 开发了基于规则和 CRF 的语法错误检测模块
\end{itemize}

%-------------------------------------------------------
\datedsubsection{\textbf{Annotator} \href{http://mark.17zuoye.net/}{[Link]}}{\textit{北京, 中国}}
\datedsubsection{\textit{主程序员}}{2018.02 --\ 2018.03}
\begin{itemize}
  \item 开发了带有登录系统、计费系统和质量控制机制的众包数据标注平台
  \item 前端使用 JS 和 Bootstrap, 后端使用异步 Web 框架 Tornado, 数据库使用 MongoDB
\end{itemize}

%-------------------------------------------------------
\datedsubsection{\textbf{MiniC Compiler} \href{https://github.com/seucs/compiler}{[Link]}}{\textit{南京, 中国}}
\datedsubsection{\textit{主程序员}}{2016.05 --\ 2016.06}
\begin{itemize}
  \item 用 Python 实现了 ``正则表达式 -> NFA'' 转换器
  \item 用 Python 实现了 LR(1) 分析器和相应的语义动作
\end{itemize}

%-------------------------------------------------------
\datedsubsection{\textbf{AI Course Projects} \href{https://github.com/yanshengjia/artificial-intelligence}{[Link]}}{\textit{南京, 中国}}
\datedsubsection{\textit{独立开发}}{2016.03 --\ 2016.06}
\begin{itemize}
  \item 使用了人工神经网络进行人脸识别
  \item 用 Matlab 实现了遗传算法来求解函数最值问题
  \item 用 C++ 实现了 A* 算法来解决24数码问题
  \item 用 C++ 实现 QS4 算法来解决百万皇后问题
\end{itemize}

%-------------------------------------------------------
% \datedsubsection{\textbf{Entity Linker} \href{https://github.com/acmom/entity-linker}{[Link]}}{\textit{南京, 中国}}
% \datedsubsection{\textit{组长}}{2016.03 --\ 2016.04}
% \begin{itemize}
%   \item 研究并比较了多种实体链接算法
%   \item 开发了一个实体链接系统, 能够将 Web 表格中的指称链接到 Wikipedia 中的参考实体
% \end{itemize}

%-------------------------------------------------------
% \datedsubsection{\textbf{My Minecraft} \href{https://github.com/seucs/my-minecraft}{[Link]}}{2015.09 --\ 2016.01}
% \textit{组长}
% \begin{itemize}
%   \item 用 OpenGL 开发了一个类似于 Minecraft 的小型 3D 游戏
% \end{itemize}

%-------------------------------------------------------
% \datedsubsection{\textbf{Weibo Spider} \href{https://github.com/yanshengjia/crawler/tree/master/weiboCrawler}{[Link]}}{2015.11 --\ 2015.11}
% \textit{独立开发}
% \begin{itemize}
%   \item 开发了一个多线程爬虫来从微博上爬取数据
% \end{itemize}

%-------------------------------------------------------
% \datedsubsection{\textbf{Virtual Campus} \href{https://github.com/acmom/vcampus}{[Link]}}{2015.09 --\ 2015.10}
% \textit{核心成员}
% \begin{itemize}
%   \item 开发了一个 Java 软件来管理学生信息
%   \item 增加了一些额外功能, 比如在线聊天, 数字图书馆, 网上商店等
% \end{itemize}